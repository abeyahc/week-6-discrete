\documentclass{report}

\input{preamble}
\input{macros}
\input{letterfonts}

\usepackage{mathtools}

\title{\Huge{Discrete Mathematics}\\Week 6}
\author{\huge{Abeyah Calpatura}}
\date{}

\begin{document}
\maketitle
\section*{4.5}
\subsection*{Exercises} \\
\text{Abeyah Calpatura} \\
\#1, 2, 7, 17, 22
\\

\textbf{\#1} 
\sol{ 
    \begin{align*}
        & \text{$n = 70$ and $d = 9$} \\
        & \text{$70 = 9q + r$} \\
        & \text{$q = 7$ and $r = 7$} \\
        & \text{Since $70 = 9(7) + 7$}
    \end{align*}
}

\textbf{\#2} 
\sol{
    \begin{align*}
        & \text{$n = 62$ and $d = 7$} \\
        & \text{$62 = 7q + r$} \\
        & \text{$q = 8$ and $r = 6$} \\
        & \text{Since $62 = 7(8) + 6$} \\
    \end{align*}
}

\textbf{\#7} 
\sol{
    \begin{align*}
        & \textbf{a.} \; \text{43 div 9: 4} \; \\
        & \textbf{b.} \; \text{43 mod 9: 7} \;
    \end{align*}
}

\textbf{\#17} 
\sol{ Prove directly from definitions that for every integer $n$, $n^2-n+3$ is odd. Use division into two cases: $n$ is even and $n$ is odd.
    \begin{align*}
        & \textbf{Case 1:} \; \text{$n$ is even} \\
        & \text{$n = 2k$ for some integer $k$} \\
        & \text{$n^2 - n + 3 = (2k)^2 - 2k + 3$} \\
        & \text{$n^2 - n + 3 = 4k^2 - 2k + 3$} \\
        & \text{$n^2 - n + 3 = 2(2k^2 - k + 1) + 1$} \\
        & \text{$n^2 - n + 3 = 2q + 1$} \\
        & \text{where $q = 2k^2 - k + 1$} \\
        & \textbf{Case 2:} \; \text{$n$ is odd} \\
        & \text{$n = 2k + 1$ for some integer $k$} \\
        & \text{$n^2 - n + 3 = (2k + 1)^2 - (2k + 1) + 3$} \\
        & \text{$n^2 - n + 3 = 4k^2 + 4k + 1 - 2k - 1 + 3$} \\
        & \text{$n^2 - n + 3 = 4k^2 + 2k + 3$} \\
        & \text{$n^2 - n + 3 = 2(2k^2 + k + 1) + 1$} \\
        & \text{$n^2 - n + 3 = 2q + 1$} \\
        & \text{where $q = 2k^2 + k + 1$} \\
        & \text{Therefore, $n^2 - n + 3$ is odd for every integer $n$}
    \end{align*}
}

\newpage

\textbf{\#22} 
\sol{ Suppose $c$ is any integer. If $c$ \textit{mod} $15 = 3$, what is $10c$ \textit{mod} $15$? In other words, if division of $c$ by 15 gives a remainder of 3, what is the remainder when 10c is divided by 15? Your solution should show that you obatin the same answer no matter what integer you start with. \\
    \begin{align*}
        & \text{$c$ \textit{mod} $15 = 3$} \\
        & \text{$c = 15q + 3$} \\
        & \text{$10c = 10(15q + 3)$} \\
        & \text{$10c = 150q + 30$} \\
        & \text{$10c = 15(10q + 2)$} \\
        & \text{$10c$ \textit{mod} $15 = 0$}
    \end{align*}
}

\newpage

\section*{4.6}
\subsection*{Exercises} \\
\text{Abeyah Calpatura} \\
\#2, 4, 6, 7, 10a
\\

\textbf{\#2}
\sol{
    \begin{align*}
        & \text{$\ceil{17/4} = \ceil{4.25} = 5$} \\
        & \text{$\floor{17/4} = \floor{4.25} = 4$} \\
    \end{align*}
}

\textbf{\#4}
\sol{
    \begin{align*}
        & \text{$\ceil{-32/5} = \ceil{-6.4} = -6$} \\
        & \text{$\floor{-32/5} = \floor{-6.4} = -7$} \\
    \end{align*}
}

\textbf{\#6}
\sol{ If $k$ is an integer, what is $\ceil{k}$? Why?
    \begin{align*}
        & \text{By deifintion of ceiling, k is an integer and the ceiling of an integer is itself since:} \\
        & \text{$k - 1 < k \leq k$} \\ 
        & \text{Therefore, $\ceil{k} = k$}
    \end{align*}
}

\textbf{\#7}
\sol{ If $k$ is an integer, what is $\ceil{k+\frac{1}{2}}$? Why?
    \begin{align*}
        & \text{By definition of ceiling, $k$ is an integer and the ceiling of an integer is itself since:} \\
        & \text{$k < k + \frac{1}{2} \leq k + 1$} \\
        & \text{Therefore, $\ceil{k + \frac{1}{2}} = k + 1$}
    \end{align*}
}

\textbf{\#10a}
\sol{ 
    \begin{align*}
        \textbf{i.} \; \text{n = 2050} \\
        & \text{$=\left(2050 + \floor*{\frac{2050-1}{4}} - \floor*{\frac{2050-1}{100}} - \floor*{\frac{2050-1}{400}}\right)$ \textit{mod} 7 } \\
        & \text{$=\left(2050 + 512 - 20 + 5\right)$ \textit{mod} 7} \\
        & \text{$=\left(2547\right)$ \textit{mod} 7} \\
        & \text{$= 6$} \\
        & \text{Corresponds to} \; \textbf{Saturday} \\
        \textbf{ii.} \; \text{n = 2100} \\
        & \text{$=\left(2100 + \floor*{\frac{2100-1}{4}} - \floor*{\frac{2100-1}{100}} - \floor*{\frac{2100-1}{400}}\right)$} \\
        & \text{$=\left(2100 + 524 - 20 + 5\right)$ \textit{mod} 7} \\
        & \text{$=\left(2609\right)$ \textit{mod} 7} \\
        & \text{$= 5$} \\
        & \text{Corresponds to} \; \textbf{Friday} \\
        \textbf{iii.} \; \text{n = 2004} \\
        & \text{$=\left(2004 + \floor*{\frac{2004-1}{4}} - \floor*{\frac{2004-1}{100}} - \floor*{\frac{2004-1}{400}}\right)$} \\
        & \text{$=\left(2004 + 500 - 20 + 5\right)$ \textit{mod} 7} \\
        & \text{$=\left(2609\right)$ \textit{mod} 7} \\
        & \text{$= 4$} \\
        & \text{Corresponds to} \; \textbf{Thursday} \\
    \end{align*}
}
\newpage

\section*{4.7}
\subsection*{Exercises} \\
\text{Abeyah Calpatura} \\
\#2, 4, 9b
\\

\textbf{\#2}
\sol{ Is $\frac{1}{0}$ an irrational number? Explain.
    \begin{align*}
        & \text{$\frac{1}{0}$ is not an irrational number, because division by 0 is not defined.} \\ 
        & \text{Thus, the number of $\frac{1}{0}$ does not exist, which implies that the number is not an irrational number.} \\ 
    \end{align*}
}

\textbf{\#4}
\sol{ Use proof by contradiction to show that for every integer $m$, $7m+4$ is not divisible by 7.
    \begin{align*}
        & \text{n is divisble if and only if there exists some integer k such that}
    \end{align*}
        \begin{equation}
            n = 7k
        \end{equation}
    \begin{align*}
        & \text{By the definition of divisble, there exists an integer k such that:} \\
        & \text{} \\
    \end{align*}
}

\textbf{\#9b}
\sol{
    \begin{align*}
        & \text{} \\
        & \text{} \\
    \end{align*}
}


\end{document}