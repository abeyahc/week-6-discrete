\documentclass{report}

\input{preamble}
\input{macros}
\input{letterfonts}
\usepackage{longdivision}

\usepackage{mathtools}
\setlength{\abovedisplayskip}{0pt}
\setlength{\belowdisplayskip}{0pt}
\setlength{\abovedisplayshortskip}{0pt}
\setlength{\belowdisplayshortskip}{0pt}

\title{\Huge{Discrete Mathematics}\\Week 6}
\author{\huge{Abeyah Calpatura}}
\date{}

\begin{document}
\maketitle
\section*{4.5}
\subsection*{Exercises} \\
\text{Abeyah Calpatura} \\
\#1, 2, 7, 17, 22
\\

\textbf{\#1} 
\sol{ 
    \begin{align*}
        & \text{$n = 70$ and $d = 9$} \\
        & \text{$70 = 9q + r$} \\
        & \text{$q = 7$ and $r = 7$} \\
        & \text{Since $70 = 9(7) + 7$}
    \end{align*}
}

\textbf{\#2} 
\sol{
    \begin{align*}
        & \text{$n = 62$ and $d = 7$} \\
        & \text{$62 = 7q + r$} \\
        & \text{$q = 8$ and $r = 6$} \\
        & \text{Since $62 = 7(8) + 6$} \\
    \end{align*}
}

\textbf{\#7} 
\sol{
    \begin{align*}
        & \textbf{a.} \; \text{43 div 9: 4} \; \\
        & \textbf{b.} \; \text{43 mod 9: 7} \;
    \end{align*}
}

\textbf{\#17} 
\sol{ Prove directly from definitions that for every integer $n$, $n^2-n+3$ is odd. Use division into two cases: $n$ is even and $n$ is odd.
    \begin{align*}
        & \textbf{Case 1:} \; \text{$n$ is even} \\
        & \text{$n = 2k$ for some integer $k$} \\
        & \text{$n^2 - n + 3 = (2k)^2 - 2k + 3$} \\
        & \text{$n^2 - n + 3 = 4k^2 - 2k + 3$} \\
        & \text{$n^2 - n + 3 = 2(2k^2 - k + 1) + 1$} \\
        & \text{$n^2 - n + 3 = 2q + 1$} \\
        & \text{where $q = 2k^2 - k + 1$} \\
        & \textbf{Case 2:} \; \text{$n$ is odd} \\
        & \text{$n = 2k + 1$ for some integer $k$} \\
        & \text{$n^2 - n + 3 = (2k + 1)^2 - (2k + 1) + 3$} \\
        & \text{$n^2 - n + 3 = 4k^2 + 4k + 1 - 2k - 1 + 3$} \\
        & \text{$n^2 - n + 3 = 4k^2 + 2k + 3$} \\
        & \text{$n^2 - n + 3 = 2(2k^2 + k + 1) + 1$} \\
        & \text{$n^2 - n + 3 = 2q + 1$} \\
        & \text{where $q = 2k^2 + k + 1$} \\
        & \text{Therefore, $n^2 - n + 3$ is odd for every integer $n$}
    \end{align*}
}

\newpage

\textbf{\#22} 
\sol{ Suppose $c$ is any integer. If $c$ \textit{mod} $15 = 3$, what is $10c$ \textit{mod} $15$? In other words, if division of $c$ by 15 gives a remainder of 3, what is the remainder when 10c is divided by 15? Your solution should show that you obatin the same answer no matter what integer you start with. \\
    \begin{align*}
        & \text{$c$ \textit{mod} $15 = 3$} \\
        & \text{$c = 15q + 3$} \\
        & \text{$10c = 10(15q + 3)$} \\
        & \text{$10c = 150q + 30$} \\
        & \text{$10c = 15(10q + 2)$} \\
        & \text{$10c$ \textit{mod} $15 = 0$}
    \end{align*}
}

\newpage

\section*{4.6}
\subsection*{Exercises} \\
\text{Abeyah Calpatura} \\
\#2, 4, 6, 7, 10a
\\

\textbf{\#2}
\sol{
    \begin{align*}
        & \text{$\ceil{17/4} = \ceil{4.25} = 5$} \\
        & \text{$\floor{17/4} = \floor{4.25} = 4$} \\
    \end{align*}
}

\textbf{\#4}
\sol{
    \begin{align*}
        & \text{$\ceil{-32/5} = \ceil{-6.4} = -6$} \\
        & \text{$\floor{-32/5} = \floor{-6.4} = -7$} \\
    \end{align*}
}

\textbf{\#6}
\sol{ If $k$ is an integer, what is $\ceil{k}$? Why?
    \begin{align*}
        & \text{By deifintion of ceiling, k is an integer and the ceiling of an integer is itself since:} \\
        & \text{$k - 1 < k \leq k$} \\ 
        & \text{Therefore, $\ceil{k} = k$}
    \end{align*}
}

\textbf{\#7}
\sol{ If $k$ is an integer, what is $\ceil{k+\frac{1}{2}}$? Why?
    \begin{align*}
        & \text{By definition of ceiling, $k$ is an integer and the ceiling of an integer is itself since:} \\
        & \text{$k < k + \frac{1}{2} \leq k + 1$} \\
        & \text{Therefore, $\ceil{k + \frac{1}{2}} = k + 1$}
    \end{align*}
}

\textbf{\#10a}
\sol{ 
    \begin{align*}
        \textbf{i.} \; \text{n = 2050} \\
        & \text{$=\left(2050 + \floor*{\frac{2050-1}{4}} - \floor*{\frac{2050-1}{100}} - \floor*{\frac{2050-1}{400}}\right)$ \textit{mod} 7 } \\
        & \text{$=\left(2050 + 512 - 20 + 5\right)$ \textit{mod} 7} \\
        & \text{$=\left(2547\right)$ \textit{mod} 7} \\
        & \text{$= 6$} \\
        & \text{Corresponds to} \; \textbf{Saturday} \\
        \textbf{ii.} \; \text{n = 2100} \\
        & \text{$=\left(2100 + \floor*{\frac{2100-1}{4}} - \floor*{\frac{2100-1}{100}} - \floor*{\frac{2100-1}{400}}\right)$} \\
        & \text{$=\left(2100 + 524 - 20 + 5\right)$ \textit{mod} 7} \\
        & \text{$=\left(2609\right)$ \textit{mod} 7} \\
        & \text{$= 5$} \\
        & \text{Corresponds to} \; \textbf{Friday} \\
        \textbf{iii.} \; \text{n = 2004} \\
        & \text{$=\left(2004 + \floor*{\frac{2004-1}{4}} - \floor*{\frac{2004-1}{100}} - \floor*{\frac{2004-1}{400}}\right)$} \\
        & \text{$=\left(2004 + 500 - 20 + 5\right)$ \textit{mod} 7} \\
        & \text{$=\left(2609\right)$ \textit{mod} 7} \\
        & \text{$= 4$} \\
        & \text{Corresponds to} \; \textbf{Thursday} \\
    \end{align*}
}
\newpage

\section*{4.7}
\subsection*{Exercises} \\
\text{Abeyah Calpatura} \\
\#2, 4, 9b
\\

\textbf{\#2}
\sol{ Is $\frac{1}{0}$ an irrational number? Explain.
    \begin{align*}
        & \text{$\frac{1}{0}$ is not an irrational number, because division by 0 is not defined.} \\ 
        & \text{Thus, the number of $\frac{1}{0}$ does not exist, which implies that the number is not an irrational number.} \\ 
    \end{align*}
}

\textbf{\#4}
\sol{ Use proof by contradiction to show that for every integer $m$, $7m+4$ is not divisible by 7.
\begin{align*}
        & \text{By the definition of divisble, there exists an integer k such that:} \\
\end{align*}
    \begin{align*}
        & \text{$7m + 4 = 7k$} \\
        & \text{$4 = 7k - 7m$} \\
        & \text{$4 = 7(k - m)$} \\
    \end{align*}
\begin{align*}
        & \text{Since $k - m$ is an integer, $4$ is divisible by $7$.} \\
        & \text{This is a contradiction, since $4$ is not divisible by $7$.} \\
\end{align*}
}

\textbf{\#9b}
\sol{Prove that the difference of any irrational number and any rational number is irrational.
    \begin{align*}
        & \text{Let us assume that x is an irrational number and y is a rational number such that their difference $x-y$ is rational.} \\
        & \text{By the definition of rational, there exist integers a, b, c, and d with $b \neq 0$ and $d \neq 0$ such that}
    \end{align*}
    \begin{align*}
        & \text{$x = \frac{a}{b}$} \\
        & \text{$y = \frac{c}{d}$} \\
        & \text{$x - y = \frac{a}{b} - \frac{c}{d}$} \\
        & \text{$x - y = \frac{ad - bc}{bd}$} 
    \end{align*}
    \begin{align*}
        & \text{Since $ad - bc$ and $bd$ are integers, $x - y$ is rational.} \\
        & \text{This is a contradiction, since $x - y$ is irrational.}
    \end{align*}
}

\newpage
\section*{4.8}
\subsection*{Exercises} \\
\text{Abeyah Calpatura} \\
\#5, 7 \\

\textbf{\#5}
\sol{
    \begin{align*} 
        & \text{The actual negation that should've been used is:} \\
        & \textbf{Suppose there exists an irrational number such that its cube root is rational.} \\
        & \text{The mistake is the negation that was used in the proof.} \\
    \end{align*}
}

\textbf{\#7}
\sol{ $3\sqrt{2}-7$ is rational.
    \begin{align*}
        & \text{Suppose that $3\sqrt{2}-7$ is rational.} \\ 
        & \text{By definition of rational, there exist integers a and b with $b \neq 0$ such that $3\sqrt{2}-7=\frac{a}{b}$}
    \end{align*} 
    \begin{align*}
        & \text{$3\sqrt{2}=\frac{a}{b}+7$} \\
        & \text{$3\sqrt{2}=\frac{a+7b}{b}$} \\
        & \text{$\sqrt{2}=\frac{a+7b}{3b}$}
    \end{align*}
    \begin{align*}
        & \text{Since $\frac{a+7b}{3b}$ is rational, this is a contradiction since $\sqrt{2}$ is irrational which implies that $3\sqrt{2}-7$ is irrational.} \\
    \end{align*}
}

\newpage
\section*{4.9}
\subsection*{Exercises} \\
\text{Abeyah Calpatura} \\
\#2, 3, 7, 8, 14 \\

\textbf{\#2}
\sol{
    \begin{align*}
        & \text{$deg(v_1)=1$} \\
        & \text{$deg(v_2)=5$} \\
        & \text{$deg(v_3)=4$} \\
        & \text{$deg(v_4)=4$} \\
        & \text{$deg(v_5)=1$} \\
        & \text{$deg(v_6)=3$} \\
        & \text{Total degree = 18} \\
        & \text{$9=\frac{1}{2}\cdot18$} \\
    \end{align*}
}

\textbf{\#3}
\sol{
    \begin{align*}
        & \text{$deg(v_1)=0$} \\
        & \text{$deg(v_2)=2$} \\
        & \text{$deg(v_3)=2$} \\
        & \text{$deg(v_4)=3$} \\
        & \text{$deg(v_5)=9$} \\
        & \text{Total Degree = $\sum_{i=1}^{5} deg(v_i)$} \\
        & \text{Total Degree = $0+2+2+3+9=16$} \\
        & \text{$m=\frac{\text{Total Degree}}{2}$} \\
        & \text{$m=\frac{16}{2}=8$} \\
        & \text{Graph has 8 edges}
    \end{align*}
}

\textbf{\#7}
\sol{
    \begin{align*}
        & \text{$deg(v_1)=1$} \\
        & \text{$deg(v_1)=1$} \\
        & \text{$deg(v_1)=1$} \\
        & \text{$deg(v_1)=4$} \\
        & \text{Total Degree = $\sum_{i=1}^{4} deg(v_i)$} \\
        & \text{Total Degree = $1+1+1+4=7$} \\
        & \text{Does not exist since 7 is odd. Due to the Handshaking Theorem, the total degree must be even.} \\
    \end{align*}
}

\textbf{\#8}
\sol{
    \begin{align*}
        & \text{$deg(v_1)=1$} \\
        & \text{$deg(v_1)=2$} \\
        & \text{$deg(v_1)=3$} \\
        & \text{$deg(v_1)=4$} \\
        & \text{Total Degree = $\sum_{i=1}^{4} deg(v_i)$} \\
        & \text{Total Degree = $1+2+3+4=10$} \\
        & \text{The total degree is even and since there are an even number of odd degrees, such a graph will exist.} \\
    \end{align*}
}

\textbf{\#14}
\sol{
    \begin{align*}
        \textbf{a.} \;
        & \text{total degree of the graph = $2 \cdot 1 + 5 \cdot 2 + x \cdot 3$} \\
        & \text{= $2 + 10 + 3x$} \\
        & \text{= $12 + 3x$} \\
        & \text{Total degree of the graph = $2 \cdot 15 = 30$. Since the graph has 15 edges} \\
        & \text{$12 + 3x = 30$} \\
        & \text{$3x = 18$} \\
        & \text{$x = 6$} \\
        & \text{The graph has 6 vertices of degree 3} \\
        & \text{6 people at the party knew three other people at the party.} \\
        \textbf{b.} \;
        & \text{the number of people at the party = $2+5+6 = 13$} \\
    \end{align*}
}

\newpage
\section*{4.10}
\subsection*{Exercises} \\
\text{Abeyah Calpatura} \\
\#2, 5, 12, 15, 16 \\

\textbf{\#2}
\sol{
    \begin{align*}
        & \text{$z=2$} \\ 
    \end{align*}
}

\textbf{\#5}
\sol{
    \begin{align*}
        \textbf{1st loop:} \;
        & \text{$f=2*1=2$ and $e=0+\frac{1}{2}=\frac{1}{2}$} \\ 
        \textbf{2nd loop:} \;
        & \text{$f=2*2=4$ and $e=\frac{1}{2}+\frac{1}{4}=\frac{3}{4}$} \\ 
        \textbf{3rd loop:} \;
        & \text{$f=4*3=12$ and $e=\frac{3}{4}+\frac{1}{12}=\frac{5}{6}$} \\
        \textbf{4th loop:} \;
        & \text{$f=12*4=48$ and $e=\frac{5}{6}+\frac{1}{48}=\frac{41}{48}$} \\
        &\textbf{$e=\frac{41}{48}$} \;
    \end{align*}
}

\textbf{\#12}
\sol{ Find the prime factorizations of 48 and 54. 
    \begin{align*}
        & \text{$48=2^4 \cdot 3$ $54 = 2 \cdot 3^3$} \\ 
        & \text{gcd(48,54) = $2 \cdot 6$} \\
        & \text{gcd(48,54) = 6} \\
    \end{align*}
}

\textbf{\#15}
\sol{ 10933 and 832
    \begin{align*}
        & \text{\intlongdivision{10933}{832}} \\
        & \text{gcd(10933, 832) = gcd(832, 117)} \\
        & \text{\intlongdivision{832}{117}} \\
        & \text{gcd(832, 117) = gcd(117, 13)} \\
        & \text{\intlongdivision{117}{13}} \\
        & \text{gcd(117, 13) = gcd(13, 0)} \\
        & \text{gcd(10933, 832) = 13} \\
    \end{align*}
}

\textbf{\#16}
\sol{ 4131 and 2431
    \begin{align*}
        & \text{\intlongdivision{4131}{2431}} \\
        & \text{gcd(4131, 2431) = gcd(2431, 1700)} \\
        & \text{\intlongdivision{2431}{1700}} \\
        & \text{gcd(2431, 1700) = gcd(1700, 731)} \\
        & \text{\intlongdivision{1700}{731}} \\
        & \text{gcd(1700, 731) = gcd(731, 238)} \\
        & \text{\intlongdivision{731}{238}} \\
        & \text{gcd(731, 238) = gcd(238, 17)} \\
        & \text{\intlongdivision{238}{17}} \\
        & \text{gcd(238,17) = gcd(17, 0)} \\
        & \text{gcd(4131, 2431) = 17} \\
    \end{align*}
}

\end{document}