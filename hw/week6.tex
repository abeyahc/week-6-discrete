\documentclass{report}

\input{preamble}
\input{macros}
\input{letterfonts}

\title{\Huge{Discrete Mathematics}\\Week 6}
\author{\huge{Abeyah Calpatura}}
\date{}

\begin{document}
\maketitle
\section*{4.5}
\subsection*{Exercises} \\
\text{Abeyah Calpatura} \\
\#1, 2, 7, 17, 22
\\

\textbf{\#1} 
\sol{ 
    \begin{align*}
        & \text{$n = 70$ and $d = 9$} \\
        & \text{$70 = 9q + r$} \\
        & \text{$q = 7$ and $r = 7$} \\
        & \text{Since $70 = 9(7) + 7$}
    \end{align*}
}

\textbf{\#2} 
\sol{
    \begin{align*}
        & \text{$n = 62$ and $d = 7$} \\
        & \text{$62 = 7q + r$} \\
        & \text{$q = 8$ and $r = 6$} \\
        & \text{Since $62 = 7(8) + 6$} \\
    \end{align*}
}

\textbf{\#7} 
\sol{
    \begin{align*}
        & \textbf{a.} \; \text{43 div 9: 4} \; \\
        & \textbf{b.} \; \text{43 mod 9: 7} \;
    \end{align*}
}

\textbf{\#17} 
\sol{ Prove directly from definitions that for every integer $n$, $n^2-n+3$ is odd. Use division into two cases: $n$ is even and $n$ is odd.
    \begin{align*}
        & \textbf{Case 1:} \; \text{$n$ is even} \\
        & \text{$n = 2k$ for some integer $k$} \\
        & \text{$n^2 - n + 3 = (2k)^2 - 2k + 3$} \\
        & \text{$n^2 - n + 3 = 4k^2 - 2k + 3$} \\
        & \text{$n^2 - n + 3 = 2(2k^2 - k + 1) + 1$} \\
        & \text{$n^2 - n + 3 = 2q + 1$} \\
        & \text{where $q = 2k^2 - k + 1$} \\
        & \textbf{Case 2:} \; \text{$n$ is odd} \\
        & \text{$n = 2k + 1$ for some integer $k$} \\
        & \text{$n^2 - n + 3 = (2k + 1)^2 - (2k + 1) + 3$} \\
        & \text{$n^2 - n + 3 = 4k^2 + 4k + 1 - 2k - 1 + 3$} \\
        & \text{$n^2 - n + 3 = 4k^2 + 2k + 3$} \\
        & \text{$n^2 - n + 3 = 2(2k^2 + k + 1) + 1$} \\
        & \text{$n^2 - n + 3 = 2q + 1$} \\
        & \text{where $q = 2k^2 + k + 1$} \\
        & \text{Therefore, $n^2 - n + 3$ is odd for every integer $n$}
    \end{align*}
}

\newpage

\textbf{\#22} 
\sol{ Suppose $c$ is any integer. If $c$ \textit{mod} $15 = 3$, what is $10c$ \textit{mod} $15$? In other words, if division of $c$ by 15 gives a remainder of 3, what is the remainder when 10c is divided by 15? Your solution should show that you obatin the same answer no matter what integer you start with. \\
    \begin{align*}
        & \text{$c$ \textit{mod} $15 = 3$} \\
        & \text{$c = 15q + 3$} \\
        & \text{$10c = 10(15q + 3)$} \\
        & \text{$10c = 150q + 30$} \\
        & \text{$10c = 15(10q + 2)$} \\
        & \text{$10c$ \textit{mod} $15 = 0$}
    \end{align*}
}

\newpage

\section*{4.6}
\subsection*{Exercises} \\
\text{Abeyah Calpatura} \\
\#2, 4, 6, 7, 10a
\\

\end{document}
